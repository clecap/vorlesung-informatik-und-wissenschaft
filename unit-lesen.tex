
% TexStudio Spellchecking Command
% !TeX spellcheck = de_DE_OLDSPELL

\unit{Lesen}{Lesen}

Lesen ist in den Wissenschaften und auch sonst eine zentrale Kulturtechnik. Mit Schnelllesen haben
wir uns bereits beschäftigt. Neben der reinen  Geschwindigkeit spielt auch die Technik des Durcharbeitens
des Lesestoffs eine wichtige Rolle. Dazu geben die folgenden Texte interessante Hinweise:

\add{Lesen (1)}
{Zentrale Studienberatung: Wissenschaftliche Lesetechniken.
Studientechniken \& Expertenbeiträge aus der Reihe \textit{Fit fürs Studium},
Universität Paderborn.}
{FILES/wissenschaftliche-lesetechniken.pdf}
{\href{https://groups.uni-paderborn.de/zsb-fit-fuers-studium/index.php/wissenschaftliches-lesen-recherchieren/}{\online}}
\pages{3}


\add{Lesen (2)}
{Zentrale Studienberatung: Die Fünf-S-Methode 
Studientechniken \& Expertenbeiträge aus der Reihe \textit{Fit fürs Studium},
Universität Paderborn.}
{FILES/die-fuenf-s-methode.pdf}
{\href{https://groups.uni-paderborn.de/zsb-fit-fuers-studium/index.php/allgemein/die-fuenf-s-methode/}{\online}}
\pages{3}

\add{Lesen (3)}
{S. Keshav: How to Read a Paper. University of Waterloo. Version 17-02-2016.}
{FILES/how-to-read-a-paper}
{\href{http://ccr.sigcomm.org/online/files/p83-keshavA.pdf}{\online}}
\pages{2}.


\add{Lesen (4)}
{Zentrum für Weiterbildung und Kompetenzentwicklung: Lesetechniken und Exzerpieren.
Hochschule Düsseldorf.}
{FILES/lesetechniken.pdf}
{\href{https://zwek.hs-duesseldorf.de/Documents/Downloadportal\%20Schreibberatung/AB\%20Lesetechniken_2019.pdf}{\online}}
\pages{3}


\add{Lesen (5)}
{F. Rost: Lern- und Arbeitstechniken für das Studium.
8. Auflage.
Springer, 2018.
Kapitel 9: Wissenschaftliche Texte lesen, verstehen und verarbeiten. pp~191-227.}
{FILES/rost-lesen.pdf}
{}
\pages{37}

Die Auswahl in diesem Abschnitt verweist auf ein weiteres Thema: Zu einer Thematik gibt es oft
viele Texte: Kurze und lange Texte; Texte, die sich ergänzen; Texte, die sich selber oder anderen
widersprechen. Die Herausforderung ist nun, sich rasch einen Überblick zu verschaffen:
Was ist wichtig? Was ist weniger wichtig? Warum?

Wir müssen also nicht immer jeden Text Wort-für-Wort lesen -- aber wir müssen uns über viele
Texte einen Überblick verschaffen.

Könnten ChatGPT und ähnliche Werkzeuge uns die Arbeit des Lesens abnehmen?
Sicherlich sind von solchen Tools noch spannende Hilfestellungen zu erwarten.
Gleichwohl bleiben wir selber für die Texte verantwortlich, die wir
mit Hilfe solcher Systeme erarbeiten. Die folgenden Überlegungen
schildern einiger der engen Grenzen dieser Hilfsmittel:

\add{Bemerkungen zu ChatGPT (1)}
{C. H. Cap: ChatGPT: Läuft die Diskussion richtig? Noch unveröffentlichtes Manuskript.}
{FILES/cap-chatgpt.pdf}
{}
\pages{3}


\add{Bemerkungen zu ChatGPT (2)}
{C. H. Cap: Danke, ChatGPT. Kommentar zu ChatGPT, auf verschiedenen sozialen Netzen gepostet.}
{FILES/danke-chat-gpt.pdf}
{}
\pages{2}

