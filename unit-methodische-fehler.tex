
% TexStudio Spellchecking Command
% !TeX spellcheck = de_DE_OLDSPELL



\unit{Methodische Fehler}{Methodische Fehler}

\paragraph{Richtige Regeln:}
Regeln anzugeben, die nur abgearbeitet werden müssen und die dann zu \enquote{richtigen}
und gleichzeitig nicht banalen
wissenschaftlichen Ergebnissen
führen, ist schwer bis unmöglich -- wir erinnern uns an den Vortrag von
\textsc{Richard Feynman} über \textit{Cargo Cult Wissenschaft}.

\paragraph{Falsche Regeln:}
Viel einfacher wäre es, Regeln anzugeben,
die fehlerhafte Resultate erzeugen -- und deshalb
wollen wir das hier jetzt auch tun!
Natürlich wollen wir diese falschen Regeln nicht anwenden, aber wir sollten sie
aus einer ganzen Reihe von Gründen kennen.


\paragraph{Kognitive Verzerrungen:}
Das menschliche Gehirn unterliegt einer sehr großen Anzahl kognitiver Verzerrungen.
Manche dieser Verzerrungen können evolutionär erklärt werden; einige bringen in bestimmten
Umweltsituationen Überlebensvorteile; andere wiederum können zu Fehlentscheidungen führen.
Der Mensch ist nicht das rationale Wesen, als das er sich selber manchmal beschreibt.
Wenn wir etwas über diese Mechanismen wissen, dann können wir Fehler im eigenen Denken leichter erkennen.

\paragraph{Abgrenzung zu Manipulation:}
Bereits im Altertum kannte die Kunst der Rhetorik viele Gefühls- und Denkfiguren,
um Menschen von etwas zu überzeugen oder sie zu bestimmten Handlungen zu veranlassen. 
Im Zeitalter der wissenschaftlichen Psychologie sowie
digitaler Medien wurden diese Ansätze perfektioniert. Wir begegnen fehlerhaften Denkfiguren im öffentli\-chen
und medialen Raum daher relativ häufig.
Die Grenzen zwischen Wissenschaft und Pseudowissenschaft, zwischen Einsicht, Propaganda und Manipulation 
sind dabei oft fließend.


\paragraph{Falsifikation:}
Wissenschaft sucht nicht Wahrheiten sondern stellt Thesen auf und falsifiziert diese.
Eine Einführung dazu finden Sie in 
der Lehreinheit \hyperlink{Digitalvorlesung}{Wissenschaft im Fallbeispiel der Digitalvorlesung},
die das anhand von einfachen Fallbeispielen erläutert.
Wenn wir also ohnehin keine Regeln haben, um \enquote{Wahrheit} herauszufinden, dann sollten wir
als Wissenschaftler also möglichst gut darin werden, Fehler zu finden.

\paragraph{Grenzen der Modelle:}
An Fehlern sind wir natürlich nicht deshalb interessiert,
weil Fehler besonders erstrebenswert wären! Wir müssen gleichwohl bedenken: Fast jede These in den Wissenschaften ist früher
oder später als falsch erkannt worden -- und war damit Quelle wissenschaftlichen Fortschritts. 
Bei jenen Theorien, die heute als \enquote{richtig} gehandelt werden,
kennen wir eben die Fehler noch nicht.

In \textit{zweiter Iterationsstufe} geht es dann darum, herauszuarbeiten,
wo die \textit{Grenzen der sinnvollen Anwendbarkeit eines Modells liegen}. Kaum eine Theorie gilt universell für alle
Anwendungsbereiche!
Auch in Ihrer Bachelorarbeit sollten Sie sich diese Frage stellen, wenn sie beim gewählten Thema adäquat ist: 
Wo enden die Anwendungsbereiche meiner Modelle und Begriffe?
Unter welchen Randbedingungen versagen die von mir benutzten Konzepte? 

\begin{figure}
\begin{center}
\scalebox{0.15}{
\includegraphics{FILES/cognitive.png}
}
\end{center}
\caption[Übersicht zu kognitiven Verzerrungen]{Eine Übersicht zu kognitiven Verzerrungen.
\href{https://upload.wikimedia.org/wikipedia/commons/6/65/Cognitive_bias_codex_en.svg}{Hier} finden Sie eine klickbare Version für Ihren Web-Browser, die zu jeder einzelnen Verzerrung die Beschreibung liefert.
\scriptsize
Quelle: John Manoogian III,
Wikimedia \url{https://upload.wikimedia.org/wikipedia/commons/6/65/Cognitive_bias_codex_en.svg},
Hier reproduziert nach CC Attribution-Share Alike 4.0 International.
}
\end{figure}

%%%%%%%%%%%%%%%%%%%%%% https://thedecisionlab.com/biases-index


\add{Kognitive Verzerrungen (1)}
{Klickbare Visualisierung über kognitive Verzerrungen}
{}
{\href{https://upload.wikimedia.org/wikipedia/commons/6/65/Cognitive_bias_codex_en.svg}{\online}}

\pagebreak

Als angehender Wissenschaftler sollten Sie (mindestens) die folgenden kognitiven Verzerrungen,
statistischen und psychologischen Effekte sowie Paradoxa kennen, die wir hier ganz knapp beschreiben:

\begin{enumerate}
\item \label{biases}\textbf{Confirmation Bias (Bestätigungsfehler):} Menschen neigen dazu, Informationen so auszuwählen und zu
interpretieren, daß sie ihre Erwartungen bestätigen.
\item \textbf{Selection Bias (Auswahlfehler):} Wird in einer Gesamtheit vorselektiert statt zufällig ausgewählt,
so verzerrt das die Ergebnisse. 
\item \textbf{Survivor Bias:} Wenn erfolgreiche Situationen stärker sichtbar sind oder im Extremfall nur erfolgreiche
Situationen überhaupt in die Beobachtung aufgenommen werden, dann werden Erfolgswahrscheinlichkeiten systematisch überschätzt.
\item \textbf{Ankereffekt:} Einschätzungen werden von Reizen beeinflußt, die mit der Schätzaufgabe nichts zu tun haben, aber zum
Zeitpunkt der Schät\-zung im Gehirn aktiviert wurden.
\item \textbf{Fundamentaler Attributionsfehler:} Neigung von Beobachtern, 
  den Einfluß von Personen auf einer Ergebnis zu überschätzen und den Einfluß externer Faktoren zu unterschätzen. 
\item \textbf{Selbstwertdienliche Attributionstendenz:} Neigung, positive Resultate auf die eigene Leistung und
negative Resultate auf externe Faktoren zurückzuführen.
\item \textbf{Backfire-Effekt:} Erfährt ein Mensch neue Fakten, die seinen Ansichten widersprechen, so können diese neuen
Fakten seine Ansichten sogar noch verstärken.
\item \textbf{Dunning-Kruger Effekt:} Menschen mit geringen Fähigkeiten über\-schät\-zen oft ihre Fähigkeiten zur Lösung einer Aufgabe.
\item \textbf{Availability Bias:} Menschen neigen dazu, die Wahrscheinlichkeiten einer Situation höher einzuschätzen, die sie
sich leichter vorstellen kön\-nen, von der sie gelesen oder gehört haben oder die sie bereits einmal selber erlebt haben.
\item \textbf{Halo Effekt:} Tendenz, bei einer Person von der Qualität einer bekannten Eigenschaft auf die Qualität einer
unbekannten, damit nicht zusammenhängenden Eigenschaft zu schließen.
\item \textbf{Simpsonsches Paradoxon:} Unterteilungen einer Gesamtheit in Gruppen können in den Gruppen Korrelationen erzeugen, die in der Gesamtheit
nicht oder anders vorhanden sind.
\item \textbf{Berksonsches Paradoxon:} Vorauswahlen können Korrelationen erzeugen, die in der Grundgesamtheit nicht bestehen.
\item \textbf{Versuchspersonen- oder Placebo-Effekt:} Die Erwartungshaltung einer Versuchsperson beeinflußt das von dieser Versuchsperson berichtete Ergebnis. 
\item \textbf{Versuchsleiter- oder Rosenthal-Effekt:} Die Erwartungshaltung des Versuchsleiters beeinflußt das Ergebnis
des von ihm angeleiteten Versuchs.
\item \textbf{Versuchs- oder Hawthorne-Effekt:} Das Wissen um die Tatsache eines Versuchs beeinflußt das Ergebnis des Versuchs.
\item \textbf{Sunk Cost Fallacy:} 
Aktivitäten, die sich eigentlich bereits als wenig hilfreich erwiesen haben, werden fortgesetzt, allein
deshalb, weil bereits viel in sie investiert wurde.
\item \textbf{Planning Fallacy:} Tendenz, die Zeit, Kosten und Risiken künftiger Handlungen zu unterschätzen und die Vorteile
dieser Handlungen zu überschätzen.
\end{enumerate}



\paragraph{Benutzerschnittstellen:}
Benutzerschnittstellen sind ein beliebtes An\-wen\-dungs\-feld für manipulative psychologische Techniken.
Für Informatiker gibt es dazu einen spannenden Kurzvortrag und eine Webseite, die Sie für die
Thematik sensibilisiert:


\add{Dark Patterns (1)}
{Kurzvortrag über Dark Patterns im Web}
{}
{\href{https://youtu.be/Zmv7cRELFtM}{\online}}
(Dauer 6:50)


\add{Dark Patterns (2)}
{Katalog von Dark Patterns}
{}
{\href{https://www.darkpatterns.org/types-of-dark-pattern}{\online}}.

\add{Persuasive Technology}
{Center for Humane Technology: Persuasive Technology: How does technology use design to influence my behavior?}
{FILES/persuasive-technology.pdf}
{\href{https://www.humanetech.com/youth/persuasive-technology}{\online}}
\pages{20}

\add{Attention Economy}
{Center for Humane Technology: The Attention Economy: Why do tech companies fight for our attention?}
{FILES/the-attention-economy-issue-guide.pdf}
{\href{https://www.humanetech.com/youth/the-attention-economy}{\online}}
\pages{14}

\add{Social Media}
{Center for Humane Technology: Take Control of Your Social Media Use: How do I shift my use of social media for good?}
{FILES/social-media-use.pdf}
{\href{https://www.humanetech.com/youth/take-control-of-your-social-media-use}{\online}}
\pages{11}
