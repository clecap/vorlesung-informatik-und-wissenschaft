

% TexStudio Spellchecking Command
% !TeX spellcheck = de_DE_OLDSPELL


\unit{Schnellesen}{Schnellesen}

Oops! Da kommt ja eine ganze Menge Lesestoff auf mich zu!! Wie bewältige ich das bloß?
\textit{Dazu} empfehle ich Ihnen das folgende unterhaltsame Video.\label{video}


\add{Speed Reading}
{Puls Reportage: Doppelt so schnell lesen in nur einer Woche?}
{}
{\href{https://www.youtube.com/watch?v=uRO4c0fCF9o}{YouTube Video: https://www.youtube.com/ watch?v=uRO4c0fCF9o (Dauer 14:28)}}

\textit{Jetzt} das Video ansehen -- \textit{dann erst} weiterblättern!

\newpage


\paragraph{Eigenbeobachtung zum Behalten:}
Sie haben sich das Video angesehen? Gut.
Vorsicht Falle: Glauben Sie bloß nicht, daß Sie es durch reines \textit{Ansehen} auch
verinnerlicht haben. Wenn Sie sich an die Anweisungen gehalten haben\footnote{nämlich erst
das Video anzusehen und erst danach weiterzublättern}, so könnte Sie der folgende Test
überzeugen. Vermutlich haben Sie nur wenig rezipierend zugesehen. Erinnern Sie
sich an die Selbsterkenntnis der jungen Dame im Video?

Die Testfragen an Sie lauten nun:
\begin{enumerate}
\item Wie viele Tipps wurden in dem Video gegeben?
\item Wie lauteten diese Tipps genau?
\end{enumerate}

Wenn Sie das jetzt nicht beantworten können, dann war die Zeit offenbar umsonst.
Sie sollten sich das Video gegebenenfalls also nochmal ansehen... \frownie{}. \textit{Davor} will ich Sie
aber noch auf etwas weiteres aufmerksam machen und Ihnen eine Beobachtungsaufgabe für die Betrachtung
mitgeben.

\paragraph{Meta-Ebene:}

Auf der Meta-Ebene ergeben sich aus dem Video für die aufmerksamen Zuseher 
viele weitere wichtige Erkenntnisse: 
\begin{enumerate}
\item \textbf{Wissensstand:} Es ist wichtig, sich mit dem Wissensstand zu beschäfti\-gen, den es zu einer
Sache bereits gibt. Der muß nicht immer stimmen, erleichtert aber den 
Einstieg\footnote{Wie wir den Einsteig meistern, erfahren wir in der 
entsprechenden Lerneinheit der Digitalvorlesung über 
wissenschaftliches Arbeiten mit Literatur.}.
\item \textbf{Vorgehensweise:} 
Wissenschaft hat viel mit \textit{Messen}, \textit{Operationalisieren} und \textit{Empirie} zu 
tun\footnote{Mehr dazu in der entsprechenden Lerneinheit der Digitalvorlesung.}.
\item \textbf{Einstellung:} Wesentlich sind \textit{Ausdauer} und \textit{Beharrlichkeit}.
\item \textbf{Überprüfen:} Wichtig ist, den eigenen Wissensstand immer wieder kritisch zu hinterfragen und
zu überprüfen.
\item \textbf{Effizienz:} Es gibt nur endlich viel Zeit. Lesen lernen also ... in einer Woche.
\end{enumerate}


\paragraph{Beobachtungsaufgabe:} Wenn Sie sich das Video jetzt gleich noch einmal ansehen, achten Sie bitte
auf die folgenden Aspekte. Machen Sie sich dazu Notizen. Behalten Sie die Beobachtungsaufgaben im Blick.
Falls notwendig, halten Sie das Video kurz an.

\textbf{4 Fragen:}
\begin{enumerate}
\item Wie viele Tipps gibt das Video?
\item Wie lauten diese Tipps?
\item Welche Stellen im Video könnten denn auf die oben erwähnten 5 \enquote{wichtigen Erkenntnisse auf
der Meta-Ebene}
anspielen?
\item Wie schnell kann man in einer Woche lesen lernen?
\end{enumerate}


Weitere Hinweise zum Schnelllesen finden Sie in dem folgenden Text. Vergleichen Sie die
Empfehlungen mit jenen aus dem Video!




\add{Schnellesen}
{D. Berger-Grabner: Wissenschaftliches Arbeiten in den Wirtschafts- und Sozialwissenschaften.
3. Auflage. Springer-Gabler, 2016.
Abschnitt 2.4: Speed Reading.
pp 47-55.}
{FILES/berger-grabner-speed-reading.pdf}
{}
\pages{9}

%%%%%%%%%%%%%%%%%%%%%%%%


Das folgende Merkblatt stammt zwar aus einer anderen Wissenschaft, näm\-lich der Germanistik, 
enthält aber eine ganze Reihe sehr guter Hinweise:


\add{Merkblatt Lesetechniken}
{H. Siebenpfeiffer; Lesetechniken. Merkblatt des Instituts für Germanistik der Universität Greifswald.}
{FILES/Merkblatt-4-Lesetechniken.pdf}
{\href{https://germanistik.uni-greifswald.de/storages/uni-greifswald/fakultaet/phil/germanistik/Mitarbeitende/Siebenpfeiffer/Merkblatt__4_Lesetechniken.pdf}{\online}}
\pages{4}









