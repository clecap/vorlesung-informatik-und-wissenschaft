
% TexStudio Spellchecking Command
% !TeX spellcheck = de_DE_OLDSPELL

\unit{Wissenschaftliche Literatur}{Wissenschaftliche Literatur}

Wir haben in der Lerneinheit über wissenschaftliches Arbeiten mit Literatur erfahren, warum es so wichtig
ist, mit Literatur richtig umzugehen: 

In der Praxis findet sich nun eine große Bandbreite von Vorgehensweisen und praktischen Vorschlägen.
Wir wollen uns einige davon näher ansehen:

\add{Zitieren (1)}
{Hochschule Ludwigshafen am Rhein: Checkliste zum richtigen Zitieren.}
{FILES/checkliste-zum-richtigen-zitieren}
{\href{https://www.hwg-lu.de/fileadmin/user_upload/service/studium-und-lehre/Schreiblabor/Checkliste_zum_richtigen_Zitieren.pdf}{\online}}
\pages{1}

\add{Zitieren (2)} 
{H. Meyer, S. E. Buchanan: Universität Konstanz: Checkliste korrektes Zitieren.
Oktober 2021.}
{FILES/checkliste-korrektes-zitieren.pdf}
{\href{https://www.uni-konstanz.de/typo3temp/secure_downloads/56005/0/aa6c4978a124f4750b3f880768fa67be473965e6/Checkliste_Korrektes_Zitieren.pdf}{\online}}
\pages{1}

\add{Zitieren (3)}
{Institut für Publizistik: Zitieren gemäß APA (7$^{\mbox{th}}$ Edition) Kurz-Manual.}
{FILES/zitieren-nach-apa.pdf}
{\href{https://www.studium.ifp.uni-mainz.de/files/2020/12/APA7_Kurz-Manual.pdf}{\online}}
\pages{16}

\add{Zitieren (4)}
{RWTH Aachen: Richtlinien für die Zitierweise.}
{FILES/zitierrichtlinien.pdf}
{\href{https://embedded.rwth-aachen.de/lib/exe/fetch.php?media=lehre:zitierrichtlinien.pdf}{\online}}
\pages{5}

Damit kommen wir zur wichtigen praktischen Frage, die Sie sich sicher gerade stellen.
Wie soll \textit{ich} nun in meiner Bachelor- und in meiner Masterarbeit und in später in Firmenberichten
oder in wissenschaftlichen Arbeiten zitieren?

\textbf{Fall 1:} Es gibt eine Vorgabe\footnote{An dem Lehrstuhl, an dem ich schreibe; bei der Zeitschrift, in der ich veröffentlichen
will; bei der Firma, in der ich einen Bericht erstelle usw.}. Dann halte ich diese Vorgabe ein.

\textbf{Fall 2:} Es gibt keine Vorgabe. Dann weiß ich nun, welche Hintergründe die unterschiedlichen Vorgaben
verfolgen. Ich nutze dann eine Zitierweise, die mir sinnvoll erscheint, und halte diese in der ganzen Arbeit
durchgängig ein.

