
% TexStudio Spellchecking Command
% !TeX spellcheck = de_DE_OLDSPELL

\unit{Selbstorganisation und Zeitplanung}{Selbstorganisation und Zeitplanung}

\paragraph{Organisationsaufgabe Reader:}
Wie bekommen wir den Reader zeitlich auf die Reihe? Mal sehen...

\paragraph{Das Mengengerüst:}
Der Reader umfaßt \crtrefnumber{totalPageNumber}\footnote{Das hat \LaTeX\ selber berechnet
aufgrund der eingebundenen Texte.} Textseiten.\footnote{Das Dokument selber
ist deutlich länger, da ich bei den Büchern auch die Titelei und die Inhaltsverzeichnisse eingebunden habe.
Das kann Ihnen bei der Entscheidung helfen, ob Sie vielleicht noch in andere Kapitel der jeweiligen Texte
hineinlesen wollen. Die meisten Bücher sind übrigens an der UB Rostock vorhanden.} Eine Normseite ent\-hält rund 250 Wörter.\footnote{vgl.\ Wikipedia: \url{https://de.wikipedia.org/wiki/Normseite}, abgerufen am 16.\ März 2022.}
Wir unterstellen eine Lesegeschwindigkeit von einer Normseite pro Minute.\footnote{Wir erinnern uns an das
\href{https://www.youtube.com/watch?v=uRO4c0fCF9o}{YouTube Video: https://www.youtube.com/ watch?v=uRO4c0fCF9o (Dauer 14:28)}
über Schnellesen, das wir in \useunit{Schnellesen} angesehen haben. 
} Das wären also 60 Seiten pro Stunde.
Wir wollen auch einmal Pause machen, Notizen anfertigen, Dinge durchdenken und
unterstreichen.
Gehen wir also von 30 Seiten pro Stunde aus!

\paragraph{Was schaffen wir?}
Da es immerhin um wissenschaftliche Arbeitstechnik geht, also das \textit{wichtigste} Handwerkzeug für unser
Berufsleben,
nehmen wir uns vor, pro Woche 2 Stunden Zeit mit diesem Reader zu verbringen.
Wann könnten wir das einplanen? Sagen wir Montag und Donnerstag abends jeweils
eine Stunde.
Ein Semester hat 15 Vorlesungswochen und danach noch Zeit zur
Vorbereitung auf Prüfungen. Wir rechnen erst einmal nur mit der Vorlesungszeit.
In dieser bekommen wir bei der angenommenen Planung
$2 * 15 * 30 = 900$ Seiten gelesen!
Das ist eine ganze Menge -- und deutlich mehr als dieser Reader umfaßt!

\paragraph{Zeitbudget und Arbeitsdisziplin:}
Wir haben in Abschnitt \ref{arbeitsaufwand} mit 17 Stunden Arbeitsaufwand für den Reader gerechnet.
Nun verwenden wir erneut die \textit{halbe} Leseleistung von der oben angeführten üblichen Lesegeschwindigkeit:
Das waren 30 Seiten pro Stunde.
Im Zeitbudget des Moduls bekommen wir also 510 Seiten gelesen -- das ist ganz deutlich mehr Seiten, als der Reader umfaßt.

Der Sinn dieser Rechnung ist nun \textit{nicht} die Erbsenzählerei, wie viele Seiten wir \textit{noch} in unsere Köpfe hineingestopft bekommen.
Es geht hier um die Beobachtung, daß wir bei stringenter Planung und ein wenig Arbeitsdisziplin durchaus sehr viel
erreichen können. Voraussetzung dafür \textit{ist} aber die vorgängige Zeitplanung. Ohne diese Überlegung hätten wir 
möglicherweise aus Schreck vor dem Vorhaben kapituliert.


\paragraph{Alles durchplanen?}
Man muß nun nicht alles bis auf die letzte Minute durchplanen -- aber eine
gewisse Kenntnis von Zeit- und Selbstmanagement-Techniken kann hilfreich sein!
Dabei kommt es nicht so sehr auf die spezifische Technik an, die Sie nutzen, 
auf das Konzept, die Software oder die App.
Wichtig ist nur, \textit{daß} wir einen Plan aufstellen und über die Frage
von Selbstorganisation und Zeitplanung reflektieren.

\paragraph{Was ist das Wichtigste?}
Wir sehen, daß wir bereits mit einem über\-schau\-ba\-ren Zeitaufwand bemerkenswerte Ziele erreichen können, wenn wir
\textit{regelmäßig} vorgehen:
$900$ Seiten lesen, am verregneten Wochenende unmittelbar vor der Klausur?
Das wird nichts! Aber an zwei Tagen die Woche durch das ganze Semester hindurch? Das klappt!
Wir erinnern uns auch an die Bedeutung der Wiederholung in der Lerntheorie
und an die Gedächtniskurve von \textsc{Ebbinghaus}!\footnote{Wenn nicht, dann
hier nachsehen: \url{https://de.wikipedia.org/wiki/Vergessenskurve}.}.

Kleine Tricks können uns das Leben viel leichter machen!

Daher ist es sinnvoll, sich ein wenig mit Arbeitstechniken zu beschäftigen.

\add{Organisation und Zeit (1)}
{C. Stickel-Wolf, J. Wolf: Wissenschaftliches Arbeiten und Lerntechniken.
4. Auflage. Gabler-Verlag Wiesbaden, 2006.
Kapitel 5: Studienorganisation
pp. 335-364.}
{FILES/stickel-wolf-erfolgreich-studieren-kapitel-5-studienorganisation.pdf}
{}
\pages{30}

\add{Organisation und Zeit (2)}
{D. Berger-Grabner: Wissenschaftliches Arbeiten in den Wirtschafts- und Sozialwissenschaften.
3. Auflage. Springer-Gabler, 2016.
Abschnitt 2.4: Speed Reading.
pp. 33-47.}
{FILES/berger-grabner-tools.pdf}
{}
\pages{15}


Was wird denn sonst noch so alles von mir erwartet? Dazu gibt der folgende Text eine kurze Übersicht:

\add{Arbeitstechniken}
{F. Rost: Lern- und Arbeitstechniken für das Studium.
8. Auflage.
Springer, 2018.
Ausschnitte aus Kapitel 1.
pp 3-8.}
{FILES/rost-lern-und-arbeitstechniken-fuer-das-studium-abschnitt-aufgaben.pdf}
{}
\pages{6}







