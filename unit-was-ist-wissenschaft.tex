
% TexStudio Spellchecking Command
% !TeX spellcheck = de_DE_OLDSPELL

\unit{Was ist Wissenschaft?}{Was ist Wissenschaft?}

Zur Frage, was Wissenschaft ist und wie sie von nichtwissenschaftlichen Aktivitäten abgegrenzt werden kann,
gibt es einen bemerkenswerten Vortrag vom Nobelpreisträger für Physik, \textsc{Richard Feynman}.

\add{Cargo Cult Wissenschaft}
{R. Feynman: Sie belieben wohl zu scherzen, Mr. Feynman. Abenteuer eines neugierigen Physikers.
10. Auflage, Piper Verlag München, 2000.
Abschnitt: Cargo-Kult-Wissenschaft, pp. 448-460.}
{FILES/feynman-cargo-cult-wissenschaft.pdf}
{}
\pages{13}

\add{Cargo Cult Science}
{R. Feynman: Cargo Cult Science. Adaptation of the Caltech commencement address, 1974.}
{FILES/feynman-cargo-cult-science.pdf}
{}
\pages{6}

Der Text schildert auf eindrückliche Weise, warum die Frage
\enquote{Wie mache ich Wissenschaft?} so schwierig zu beantworten ist und weshalb es
die 7 magischen Regeln für Wissenschaft nicht gibt:
\textit{Die Flugzeuge landen dann nämlich einfach nicht...}\footnote{und um diese
Bemerkungen einsortieren zu können, müssen Sie den Text von Feynman lesen!}

Gelegentlich ist es wichtig, zu wissen, wie andere
Autoren einen Text rezipieren. Da
der Vortrag von \textsc{Feynman} so 
begriffsprägend war, wollen wir das 
hier\footnote{Sie nehmen dabei mit, daß diese
Strategie auch für Ihre Bachelorarbeit
wichtig sein kann. Für die zentralen 
Arbeiten, die Sie zitieren, sollten Sie sich kurz ansehen, wie diese rezipiert wurden:
Wer hat sie gelesen? Wer hat sie zitiert? Wie wurden sie dabei kommentiert?
Haben andere die Arbeit ähnlich interpretiert wie
Sie -- oder liegen Sie mit Ihrer Interpretation ganz daneben?} auch
einmal tun:


\add{Rezeption Cargo Cult}
{H. Hanlon: Cargo Cult Science.
European Review, Vol 21 (S1), pp. 51-55
doi:10.1017/ S1062798713000124.
\href{https://doi.org/10.1017/S1062798713000124}{DOI}}
{FILES/hanlon-cargo-cult-science.pdf}
{\href{https://www.cambridge.org/core/journals/european-review/article/cargo-cult-science/38CA581FFAB42704E5F667AA4A2D6D79}{\online}}
\pages{5}

So wundervoll diese Texte erklären, was Wissenschaft ist, so wenig helfen sie Ihnen aber \textit{praktisch} weiter,
wenn Sie selber erfolgreich Wissenschaft betreiben wollen. Deshalb enthält dieser
Reader noch weitere Inhalte. Wir werden uns auch einen Satz von Regeln für das wissenschaftliche Arbeiten erstellen. 
Wir wissen nun aber -- und das ist das Wichtige
in dieser Einheit -- warum diese Regeln, Hinweise und Methoden, so bedeutsam sie sind, allein 
aber nicht ausreichen, um die Flugzeuge zum Landen zu bringen.

Der Gedanke ist allerdings schon älter und findet sich in ähnlich prägnanter Weise,
aber weniger didaktisch aufbereitet, beim Physiker \textsc{Ernst Mach}:


\add{Wissenschaft (1)}
{Ernst Mach. Erkenntnis und Irrtum: Skizzen zur Psychologie der Forschung. Barth, Leipzig, 1906. p 200.}
{FILES/ernst-mach.pdf}
{}
\pages{1}


Wissenschaft muß von Pseudowissenschaft abgegrenzt werden. Dazu finden wir noch ein wenig
mehr bei \textsc{Lakatos}:

\add{Wissenschaft (2)}
{I. Lakatos, J. Worrall, G. Currie: The Methodology of Scientific Research Programmes.
Cambridge University Press, 1978.
Introduction: Science and Pseudoscience. pp. 1-7.}
{FILES/lakatos-science-and-pseudoscience.pdf}
{}
\pages{7}

Etwas lebhafter argumentiert dieses Video:

\add{Wissenschaft (3)}
{S. Hossenfelder: How to tell science from pseudoscience.}
{}
{\href{http://backreaction.blogspot.com/2020/06/how-to-tell-science-from-pseudoscience.html}{Video (Dauer: 8:28)}}


Natürlich hat Wissenschaft einen praktischen Nutzen:
\add{Nutzen von Wissenschaft (1)}
{B. Heesen: Wissenschaftliches Arbeiten: Methodenwissen für das Bachelor-, Master- und Promotionsstudium.
3. Auflage. Springer-Verlag Berlin Heidelberg 2014. 
Kapitel 2: Nutzen des wissenschaftlichen Arbeitens.
pp. 5-14.}
{FILES/heesen-wissenschaftliches-arbeiten-kapitel-2-nutzen-wissenschaftlichen-arbeitens.pdf}
{}
\pages{10}

Dieser Text hat zunächst einen für Geistesarbeiter sehr interessanten Inhalt, da
er Lernmythen behandelt. Zugleich aber ist er auch ein schönes Beispiel für den
kritischen Umgang mit Thesen und Mythen.
\add{Kritischer Umgang mit Mythen}
{N. Mukerji: Lernmythen. skeptiker, 2/2018, GWUP -- Gesellschaft zur wissenschaftlichen Untersuchung von Parawissenschaften e.V.}
{FILES/lernmythen.pdf}
{\href{https://www.gwup.org/zeitschrift-skeptiker}{Homepage der Zeitschrift}}
\pages{6}





