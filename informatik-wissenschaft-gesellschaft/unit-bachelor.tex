

% TexStudio Spellchecking Command
% !TeX spellcheck = de_DE_OLDSPELL


\unit{Bachelorarbeiten}{Bachelor}

\textbf{1. Perspektive:} Wissenschaftliches Arbeiten lernt man nicht, indem man die Vorgehensweisen
in der Wissenschaft \textit{emuliert, simuliert oder nachmacht}.
Dazu haben wir im ersten Teil der Lehrveranstaltung, in der Informatik und Wissenschaft (IW)
bereits zwei sehr schöne Texte gelesen. Zur Erinnerung
sind diese hier nochmals angeführt:

\add{Cargo Cult Wissenschaft}
{R. Feynman: Sie belieben wohl zu scherzen, Mr. Feynman. Abenteuer eines neugierigen Physikers.
10. Auflage, Piper Verlag München, 2000.
Abschnitt: Cargo-Kult-Wissenschaft, pp. 448-460.}
{FILES/feynman-cargo-cult-wissenschaft.pdf}
{}
\pages{13}

\add{Cargo Cult Science}
{R. Feynman: Cargo Cult Science. Adaptation of the Caltech commencement address, 1974.}
{FILES/feynman-cargo-cult-science.pdf}
{}
\pages{6}

\bigskip

\textbf{2. Perspektive:} Auf ersten Blick widersprüchlich zu dieser
ersten Perspektive erscheint nun die Überlegung, daß man eine Vorgehensweise gut
durch \textit{Beobachtung,
Nachahmung und Beispiele} erlernen kann.
Manche Philosophen und Psychologen gehen in der Tat davon aus, daß unser Leben
keine Ansammlung von Problemen ist, die rational gelöst werden können,
sondern aus einer Menge von Paradoxa besteht, die wir managen müssen.\footnote{In der
hier dargestellten Form 
stammt der Gedanke aus 
einem Vortrag der Psychotherapeutin \textsc{Esther Perel},
\textit{The Other AI: Artificial Intimacy}.
\href{https://www.youtube.com/watch?v=vSF-Al45hQU}{https://www.youtube.com/watch?v=vSF-Al45hQU}.
\color{red}\bf Diese Fußnote gibt Ihnen wiederum ein Beispiel, wie man
(1) fremde Gedanken zitiert, für die man nicht den Anspruch eigener Autorenschaft
und Originalität erhebt, und (2) wie man das tun kann, wenn es dafür
kein typisches Literaturzitat gibt. Da das ein wichtiges Beispiel ist, wird es hier aus 
\underline{didaktischen}
Gründen rot und fett hervorgeheben. In einer \underline{wissenschaftlichen}
Arbeit wäre ein solches Vorgehen
wieder weniger üblich.
} Daher seien hier nun vier Beispiele von Bachelorarbeiten deutscher Universitäten
angegeben. Die Beispiele zeigen auch die unterschiedlichen Stile,
die bei Bachelorarbeiten möglich sind. Daher werden auch Bachelorarbeiten der Universität Rostock 
von verschiedenen Lehrstühle verschieden betreut und verschieden aussehen.
Den \textit{einen allein richtigen Stil} wird es wohl nicht geben.

\bigskip

\add{Beispiel eine Bachelorarbeit von der TU Dresden}
{T. Ludyga: Entwicklung und Untersuchung eines Konzepts zur Nutzung von
Indoor-Positioning-Technologie in Systemen für adaptive, mobile Informationsbereitstellung.
Bachelorarbeit Technische Universität Dresden, 2021.
}
{FILES/ludyga.pdf}
{\href{https://tud.qucosa.de/api/qucosa\%3A77414/attachment/ATT-0/}{https://tud.qucosa.de/api/qucosa\%3A77414/attachment/ATT-0/}}
\pages{56}



\add{Beispiel einer Bachelorarbeit von der RWTH Aachen}
{N. Saritas: Integration von Feedback zur Optimierung von
hybriden Lernformen im Informatikunterricht.
Bachelorarbeit RWTH Aachen, 2021.
}
{FILES/saritas.pdf}
{\href{https://publications.rwth-aachen.de/record/841006/files/841006.pdf}{https://publications.rwth-aachen.de/record/841006/files/841006.pdf}}
\pages{124}



\add{Beispiel einer Bachelorarbeit von der Universität Heidelberg}
{J. M. Kunkel: Performance Analysis of the PVFS2 Persistency Layer.
Bachelorarbeit Universität Heidelberg, 2006
}
{FILES/kunkel.pdf}
{\href{https://archiv.ub.uni-heidelberg.de/volltextserver/6330/1/PerformanceAnalysis.pdf}{https://archiv.ub.uni-heidelberg.de/volltextserver/6330/1/PerformanceAnalysis.pdf}
}\newline
\pages{106}



\add{Beispiel einer Bachelorarbeit von der Universität Bayreuth}
{C. Brunner: Erweiterung eines channelbasierten Task-Laufzeitsystems um eine
OpenMP-Nutzerschnittstelle. Bachelorarbeit Universität Bayreuth, 2022.
}
{FILES/brunner.pdf}
{%
\href{https://epub.uni-bayreuth.de/id/eprint/6047/1/brunner2022taskopenmp.pdf}{https://epub.uni-bayreuth.de/id/eprint/6047/1/\allowbreak\relax brunner2022taskopenmp.pdf}
}
\pages{72}




%%%%%%%%%%%%%%%%%%%%%%%%










