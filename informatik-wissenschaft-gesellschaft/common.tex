%%
%% common.tex  Macro Datei für den IW Reader
%%

\pdfoptionpdfminorversion=7                        % Anpassen PDF Version an die PDFs 
                                                   % die im reader eingebunden werden
%%
%% Import Packages
%%
\usepackage[german]{babel}                         % Sprache und Trenntabellen auswählen
\usepackage{geometry}                              % Anpassen Seitengeometrie für Einbau
                                                   % externer DPF-Dateien
                                                   
\usepackage{fontawesome}                           % Einige nette Symbole einbinden
\usepackage{dtk-logos}                             % Wir brauchen das Bibtex Logo
\usepackage{manfnt}                                % Benötigtes Symbol einbauen
\usepackage{MnSymbol,wasysym}                      % Benötigtes Symbol einbauen

\usepackage[inline]{enumitem}                      % Anpassbare Listen aktiv, inline
\usepackage{hhline}                                % Linien für farbige Tabellen
\usepackage[table]{xcolor}                         % Farbige Zellen in Tabellen
\usepackage{adjustbox}                             % Zusatz; rotierte Tabellen
\usepackage{tabto}                                 % Bessere Tabulatoren

\usepackage{fancyhdr}                              % Sinnvolle headings auf den Seiten
\pagestyle{fancy}                                  % Heading Stil aktivieren

\usepackage{qrcode}                                % QR Code setzen

\usepackage{pdfpages}                              % Importieren von PDF Dateien

\usepackage[outputdir=build,cache=false]{minted}   % Einbinden von Code mit Highlighting
                                                   % Option für build directory
                                                   % Caching aus wegen Selbstinklusion

\usepackage{graphics}                              % Grafiken einbauen
\usepackage{pgfplots}                              % Plot Funktionen
\usetikzlibrary{automata, positioning}             % Plot Libraries einbinden
\usetikzlibrary{arrows, arrows.meta}               % Plot Libraries einbinden

\usepackage{comment}                               % Teile auskommentieren
\usepackage{todonotes}                             % Notizen für den Autor

\usepackage{titletoc}                              % Partielle Inhaltsverzeichnisse

%%
%% Optik des Layout etwas nachjustieren
%%
\parindent 0pt                                     % Kein Paragrapheneinzug
\parskip   0.25cm                                  % Paragraphenabstand
\usepackage{titlesec}                              % \titlespaceing command
\titlespacing{\paragraph}{0pt}{8pt}{16pt}          % Paragraphen kompakter
\usepackage[all]{nowidow}                          % Hurenkinder, Schusterjungen vermeiden
\setlist[enumerate]{topsep=0pt}                    % Anpassbare Listen nutzen und anpassen
\setlist[itemize]{topsep=0pt}
\setlist[enumerate]*{label={(\arabic*)}}

%%
%% Neue Seite am Ende von \part vermeiden
%%
%
% Quelle: https://tex.stackexchange.com/questions/42526/
%         how-to-remove-page-break-after-part-in-report-book
%
\makeatletter
\def\@endpart{}                                   % First, modify the \@endpart macro.
\patchcmd{\part}{\null\vfil}{}{}{}                % Third, suppress vertical whitespace 
                                                  % before "Part xx" material
\makeatother

%%
%% Einstellungen für schönere Hyperlinks
%%
\usepackage{xurl}                                  % Erlaubt bessere Trennung von URLs
\usepackage[]{hyperref}                            % Hyperlinks  
                                                   % CAVE: load hyperref AFTER titlesec
\hypersetup{
  bookmarksnumbered=true,                          % number PDF
  breaklinks=true,                                 % may break links
  colorlinks,
  linkcolor={red},                                 % internal links 
  citecolor={red}, 
  urlcolor={blue},                                 % linked urls
  filecolor={blue}                                 % urls which open local files
}

\usepackage[type={CC},                             % Setup copyright notice
            modifier={by-nc-sa}, 
            version={4.0}]{doclicense}

%%
%% Einige Convenience Abkürzungen
%%
\def\PDF{\faFilePdfO PDF}
\def\online{\faExternalLink online}
\def\tikz{Ti\textit{k}Z}

%%
%% Macro, um die totale Seitenzahl des Readers zu ermitteln
%%
%
%  \pages{ <num> }    Addiert und gibt aus
%  \pages*{ <num> }   Addiert und gibt nicht aus
%  Zähler:            totalpages
%
\newcounter{totalpages}
\setcounter{totalpages}{0}  % Lokalen Zähler anlegen
\makeatletter
\def\@pages#1{\addtocounter{totalpages}{#1}}%      % Makro Seitenzahl ohne Ausgabe
\def\@@pages#1{%
  \def\one{1}%
  \def\arg{#1}%
  (#1~\ifx\one\arg{Seite}\else{Seiten}\fi)%
  \addtocounter{totalpages}{#1}}%                  % Makro Seitenzahl MIT Ausgabe
\def\pages{\@ifstar\@pages\@@pages}%
\makeatother

%%
%% Referenzen über mehr als einen LaTeX-Lauf merken
%%
%
%  Bsp: Wert von totalpages am Ende des Dokuments in totalPageNumber speichern     
%       \storereference{totalPageNumber}{\thetotalpages}
%
%       Wert von totalPageNumber an anderer Stelle nutzen                    
%       \crtrefnumber{totalPageNumber}
%
\usepackage{crossreftools}
\crtrefundefinedtext{0}                            % Undefinierte Referenzen sind 0
\newcommand{\storeReference}[2]{\crtcrossreflabel*{#2}[#1]}


%%
%% \glabel definiert einen globalen label, der in \currentLabel gespeichert wird
%% Benötigt für die backlinks, wo eine Aufgabe etc benutzt wurde
%%
\def\glabel#1{\label{#1}\hypertarget{#1}{}\gdef\currentLabel{#1}}

%%
%% LATEXTRAINING
%%
%   \latexaufgabe  Setzt die Titel-Zeile für eine Aufgabe
%   \latexaufgabe{ <Bezeichnung-der-Aufgabe> }{ <label-zum-referenzieren-der-aufgabe> }
%
%   \uselatexaufgabe  Verweist auf eine Aufgabe im Text
%   \uselatexaufgabe{ <label-zum-referenzieren-der-aufgabe> }
%   Bsp: Wir bearbeiten \useaufgabe{ <label-zum-referenzieren-der-aufgabe> }
%
\def\latexaufgabe#1#2{%
  \stepcounter{latexaufgabencounter}%
  \subsection*{\phantomsection\relax \hypertarget{#2}{}#1%
     \ifdefined\instruktorflag{\hfill\color{gray}#2}\fi}%      
   \label{lt\thelatexaufgabencounter}%
  %   Referenznamen nur in Dozentenversion zeigen
  \addcontentsline{toc}{subsection}{#1}%
  \storeReference{#2}{#1}%    Speichere Bezeichnung unter dem label
}

\def\uselatexaufgabe#1{\hyperlink{#1}{\crtrefnumber{#1}}}

\newcounter{latexaufgabencounter}


%%
%% AUFGABEN
%%
%   \aufgabe  Setzt die Titel-Zeile für eine Aufgabe
%   \aufgabe{ <Bezeichnung-der-Aufgabe> }{ <label-zum-referenzieren-der-aufgabe> }
%
%   \useaufgabe  Verweist auf eine Aufgabe im Text
%   \useaufgabe{ <label-zum-referenzieren-der-aufgabe> }
%   Bsp: Wir bearbeiten \useaufgabe{ <label-zum-referenzieren-der-aufgabe> }
%
\def\aufgabe#1#2{%
  \stepcounter{aufgabencounter}%
  \hypertarget{#2}{}%        Target für \useaufgabe setzen
  \label{#2}% 
  \subsection*{\phantomsection\relax Aufgabe \theaufgabencounter:~#1%
     \ifdefined\instruktorflag{\hfill\color{gray}#2}\fi}%       
  % Referenznamen nur in Dozentenversion zeigen
  \addcontentsline{toc}{subsection}{Aufgabe \theaufgabencounter:~#1}%
  \storeReference{#2}{Aufgabe \theaufgabencounter:~#1}%     Speichere Mapping von 
                                         %  Aufgaben-Label zu Aufgabenbezeichnung
  \storeReference{AU#2}{\theaufgabencounter}%  Maps label to number
  \textbf{Behandelt:} In \nameref{\crtrefnumber{Label#2}}.
}
\def\useaufgabe#1{\hyperlink{#1}{\crtrefnumber{#1}}%
  \storeReference{Label#1}{\currentLabel}% Mapping von "Label" gefolgt 
  %         von Aufgaben-Label zu Label der Unit wo es behandelt wird
  \ifdefined\instruktorflag{\hfill\color{gray}#1} \fi%
}
\newcounter{aufgabencounter}

\def\sa#1{\hyperlink{#1}{\bf\crtrefnumber{AU#1}}}%  given a label provides a link to the aufgabe


%%
%% VORFÜHRUNGEN
%%
%
%  \demo  Setzt die 
%  \demo{ <Bezeichnung-der-Vorführung> }{ <label-zum-referenzieren-der-Vorführung> }
%  \usedemo{ <label-zum-referenzieren-der-Vorführung> }
%
\def\demo#1#2{
  \stepcounter{democounter}%
  \hypertarget{#2}{}%        Target für \usedemo setzen
  \subsection*{\phantomsection\relax Vorführung \thedemocounter:~#1%
    \ifdefined\instruktorflag{\hfill\color{gray}#2}\fi}%
  \label{vor\thedemocounter}%  vor<number> ist der label space der Vorführungen
  %  Referenznamen nur in Dozentenversion zeigen
  \addcontentsline{toc}{subsection}{Vorführung \thedemocounter:~#1}%
  \storeReference{#2}{Vorführung \thedemocounter:~#1}  
}
\def\usedemo#1{\hyperlink{#1}{\crtrefnumber{#1}}}
\newcounter{democounter}


%%
%% UNITS
%%
%   #1  Name to be written    
%   #2 Label used for referencing (needed in case of \LaTeX in title)
\newcounter{unitcounter}\setcounter{unitcounter}{0}%           Zähler für Units
\def\unit#1#2{%                                                Header für Leseeinheit setzen
  \clearpage%
  \stepcounter{unitcounter}%
  \setcounter{header}{1}%                                      Neue Nummerierung
  \lhead{Reader}%                                              Linke Header in Unit
  \rhead{Einheit \theunitcounter: #1}%                         Rechter Header in Unit
  \hypertarget{#2}{}%                                          Target setzen zur Unit
  \storeReference{#2}{Reader Einheit \theunitcounter:~#1}%
  \storeReference{Num#2}{\theunitcounter}%
  \section*{\phantomsection\relax Einheit \theunitcounter: #1%
    \ifdefined\instruktorflag{\hfill\color{gray}#2}\fi}%
    \label{unit\theunitcounter}%
  \addcontentsline{toc}{section}{Einheit \theunitcounter:~#1}%
  \gdef\unitName{#1}
}

\def\useunit#1{\hyperlink{#1}{\crtrefnumber{#1}}}% Referenziere ganze Reader Unit über Namen.

\def\uu#1{\hyperref[\crtrefnumber{#1}]{#1}}%  referenziere reader unit über namnen, zeige nummer

\newcounter{header}\setcounter{header}{1}%    Subzähler für einzelne Elemente in Einheiten

%%
%% Zur Formatierung jener Teile im Reader, die auf derselben Ebene wie die \units sind
%% die aber selber nicht Einheiten sind, und sich daher anders verhalten müssen (Links etc)
%% Bsp: Vorwort Nachwort
%%
\def\mysection#1{
  \clearpage%
  \lhead{Reader}%
  \rhead{#1}%
  \phantomsection%   \section* setzte keine hypertargets
  \section*{#1}%
  \addcontentsline{toc}{section}{#1}%
}

%%
%% Im Footer rechts überall einen Link zurück zum Inhaltsverzeichnis setzen
%%
\rfoot{\hyperlink{Inhaltsverzeichnis}{\faListOl}}% 

%%
%% Pagestlye für den Reader / PDF Import setzen
%%
%  Quelle in den Footer schreiben.
%
\fancypagestyle{pdfpages}{%                       Page style für Reader definieren
  \renewcommand{\headrulewidth}{0pt}%             Keine Trennlinie im Header, denn wir haben 
%                                                 einen Rahmen in variabler Länge je nach
%                                                 Größe der importierten Seite
  \fancyfoot[L]{\footnotesize\pdfpagesfooter}
  \fancyfoot[C]{}%                                Zentrierter Footer wäre die Seitenzahl, 
  %                                               die wir im Reader nicht brauchen
  \fancyfoot[R]{\hyperlink{Inhaltsverzeichnis}{\faListOl}\hspace*{-1cm}}%
}

\newcommand*\pdfpagesfooter{}

\newcommand{\setlayout}[1]{%               Macro um Page Style (footer) auf der 
%                                          spezifischen Seite des PDF Include zu setzen
  \thispagestyle{pdfpages}%                Auf pdfpages page style umschalten
  \gdef\pdfpagesfooter{#1}%                Parameter in globaler Variablen speichern
}


%%
%% Pagestyle für den Anhang definieren
%%
\fancypagestyle{anhang}{
  \fancyhead[L]{}
  \fancyhead[C]{Anhang}%                                
  \fancyhead[R]{}%
}





%%
%% READER-INHALT
%%
%
%  \add         Definieren ein Inhaltselement im Reader
%    #1 Header Text für den Inhalt, dient auch als Label für Verweise    
%       CAVE:   Darf keine Sonderzeichen enthalten wegen Nutzung als Label
%    #2 Text der den Inhalt beschreibt
%    #3 Filename des Inhalts, inklusive Pfad                    oder leer {}
%    #4 Ein vollständiger Link is web in der Form \href{}{}     oder leer {}
%
\makeatletter
\newwrite\appendFile
\immediate\openout\appendFile=appendFiles.lst
\def\add#1#2#3#4{%
  \hypertarget{Kom:#1}{}%                        Target für das Kommentar setzen
  \textbf{\S \theunitcounter.\theheader:~#1} #2%  
  \def\argFile{#3}%                              Parameter in Variable speichern 
  %                                              nötig für Vergleich mit leerem Macro
  \ifx\argFile\empty\else%                       Wenn File vorhanden, 
    \relax\space\hyperlink{Ank:#1}{\PDF}\fi%     Link setzen auf Ankündigungsseite
  \def\argu{#4}%
  \ifx\argu\empty\else\relax\space\argu\fi%
  \ifx\argFile\empty\else%                       Wenn File vorhanden, 
    \write\appendFile{%                          Ankündigungsseite;  File inkludieren
      \unexpanded{\hypertarget}{#3}{}%
      \unexpanded{\hypertarget}{Ank:#1}{}%       Target für die Ankündigungsseite
      \unexpanded{\rhead{}\lhead{}}%             Header der Ankündigungsseite löschen
      \unexpanded{\chead}{\footnotesize\textbf{Einheit \theunitcounter:~\unitName}% 
        \hfill \textbf{Text:}~#1}%               Header Ankündigungsseite  
      \unexpanded{\phantomsection}%
      \unexpanded{\addcontentsline}{toc}{section}{#1}%
      \unexpanded{\large}%
      \unexpanded{\renewcommand\baselinestretch{1.2}}%
      \unexpanded{\topskip0pt\vspace*{\fill}}%
      \unexpanded{\textbf}%
      {#1} \\[6pt]%
      #2   \\[6pt]%
      #4
      \unexpanded{\vspace*{\fill}}%
      \unexpanded{\normalsize}%
      \ifdefined\mittext
        \unexpanded{\includepdf[pages=-, 
                                pagecommand=\setlayout{\textbf{Quelle:}~#2}, 
                                width=\textwidth,
                                height=\textheight,
                                keepaspectratio,
                                frame]}{#3}%
      \else\par
         Den Text finden Sie aus urheberrechtlichen Gründen nur 
         in der eingeschränkten Variante dieses Dokuments oder
         online unter der Verantwortung der entsprechend verlinkten Site.
         \vfill\pagebreak
      \fi
     %
      }%
  \fi%
  \stepcounter{header}%
}
\makeatother

\usepackage{csquotes}                         % Quotes; Paket möglichst spät laden

\ifdefined\instruktorflag%                      Wenn instruktorflag gesetzt
  \usepackage[firstpageonly=true]{draftwatermark}% Watermark auf Frontseite zur Warnung
\fi
