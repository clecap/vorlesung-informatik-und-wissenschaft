

% TexStudio Spellchecking Command
% !TeX spellcheck = de_DE_OLDSPELL


\unit{Wissenschaftliches Schreiben}{Schreiben}


\add{TU Darmstadt}
{SchreibCenter am Sprachenzentrum: Wissenschaftliches Schreiben in der Informatik. 
TU Darmstadt}
{FILES/wiss-schreiben-darmstadt.pdf}
{\href{https://www.owl.tu-darmstadt.de/media/owl/responsive\_design/owl\_anleitungen\_pdf/0036\_WissSchreiben\_in\_der\_Informatik\_2020-06.pdf}{https://www.owl.tu-darmstadt.de/media/owl/responsive\_\allowbreak\relax design/owl\_anleitungen\_pdf/\allowbreak0036\_WissSchreiben\_in\_der\_Informatik\_2020-06.pdf}}
\pages{12}



\add{Uni Paderborn: Grundlagen des wissenschaftlichen Arbeitens}
{Lernzentrum Informatik: Grundlagen des wissenschaftlichen Arbeitens}
{FILES/wiss-schreiben-paderborn.pdf}
{\href{https://cs.uni-paderborn.de/fileadmin-eim/informatik/Lernzentrum/Wissenschaftlichesarbeiten.pdf}{https://cs.uni-paderborn.de/file\allowbreak\relax admin-eim/informatik/Lernzentrum/Wissenschaftlichesarbeiten.pdf}
}
\pages{14}




\add{The Science of Scientific Writing}
{G. D. Gopen und J. A. Swan: The Science of Scientific Writing. American Scientist, 
Journal of Sigma Xi, 78(6), 550-558, 1990.}
{FILES/gopen-and-swan-science-of-scientific-writing.pdf}
{}
\pages{16}





\add{Schreibprozess}
{S. Stock, P. Schneider, E. Peper, E. Moiltor (Hrsg.): Erfolgreich
wissenschaftlich arbeiten. 2. Auflage. Springer Gabler, 2018.
Kapitel 7: Schreibprozeß}
{FILES/steffen-schreibprozess.pdf}
{}
\pages{44}




\bigskip\bigskip
\textbf{Bewertung:} Wurde der Text dann geschrieben, so stellt sich natürlich die Frage,
nach welchen Kriterien er bewertet wird.
Hier spielen dann wieder die Anforderungen an eine wissenschaftliche Arbeit eine
zentrale Rolle, die wir bereits in der
Digitalvorlesung \hyperref[DV]{\bf DV5} ausführlich behandelt haben.




\add{Qualitätskriterien}
{B. Heesen: Wissenschaftliches Arbeiten: Methodenwissen für das Bachelor-, Master- und 
Promotionsstudium.
3. Auflage. Springer-Verlag Berlin Heidelberg 2014. 
Kapitel 3: Qualitätskriterien für wissenschaftliche Arbeiten.
pp. 15-29.}
{FILES/heesen-wiss-arbeiten-kapitel-3-qualitaetskriterien.pdf}
{}
\pages{15} % OK


\add{Bewertungskriterien}
{Lehrstuhl Erwachsenenbildung der Universität Jena: 
Bewertungskriterien für Haus-, Bachelor-, Master- und Magisterarbeiten.\newline}
{FILES/jena.pdf}
{https://www.eb.uni-jena.de/ebmedia/114/bewertungskriterien-wiss-arbeiten.pdf}
\pages{6} 


\add{Kriterienkatalog}
{A. Bänsch, D. Alewell: Wissenschaftliches Arbeiten. 12. Auflage. De Gruyter, 2020.
Teil C: Kriterien zur Beurteilung wissenschaftlicher Arbeiten.
pp. 125-132.
}
{FILES/baensch.pdf}
{}
\pages{8} 












