

% TexStudio Spellchecking Command
% !TeX spellcheck = de_DE_OLDSPELL


\unit{Gliederung und Aufbau}{Gliederung}


Zu Gliederung und Aufbau einer Arbeit können verschiedene Ansätze gewählt werden.
Letztlich aber läuft es aber doch immer wieder auf dasselbe Ziel hinaus.
Daher kann es hilfreich sein, verschiedene Vorschläge anzulesen.


\add{Präzisierung der Gliederung}
{D. Brauner, H. Vollmer: Erfolgreiches wissenschaftliches Arbeiten.
Verlag Wissenschaft \& Praxis, 2008.
Kapitel 10: Präzisierung und Detaillierung der Gliederung.
pp. 119-122}
{FILES/brauner-praezisierung.pdf}
{}
\pages{4} % OK, Kerntext allein gezählt


\add{Inhaltlicher Aufbau}
{H. Balzert, M. Schröder, C. Schäfer: Wissenschaftliches
Arbeiten. 2. Auflage.
Springer Nature, 2017. 6. Kapitel: Inhaltlicher Aufbau einer wissenschaftlichen Arbeit.
pp. 63-80.}
{FILES/wiss-arbeiten-balzert-inhaltlicher-aufbau.pdf}
{}
\pages{18} % OK



\add{Formaler Aufbau}
{H. Balzert, M. Schröder, C. Schäfer: Wissenschaftliches
Arbeiten. 2. Auflage.
Springer Nature, 2017. 8. Kapitel: Formaler Aufbau wissenschaftlicher Arbeiten.
pp. 95-128.}
{FILES/wiss-arbeiten-balzert-formaler-aufbau.pdf}
{}
\pages{34} % OK

\add{The Science of Scientific Writing}
{G. D. Gopen und J. A. Swan: The Science of Scientific Writing. American Scientist, 
Journal of Sigma Xi, 78(6), 550-558, 1990.}
{FILES/gopen-and-swan-science-of-scientific-writing.pdf}
{}
\pages{16}



\add{Konkretisierung der Problemstellung}
{D. Brauner, H. Vollmer: Erfolgreiches wissenschaftliches Arbeiten.
Verlag Wissenschaft \& Praxis, 2008.
Kapitel 5: Schrittweise Konkretisierung der Problemstellung.
pp. 73-80}
{FILES/brauner-konkretisierung.pdf}
{}
\pages{8} % OK, Kerntext allein gezählz









