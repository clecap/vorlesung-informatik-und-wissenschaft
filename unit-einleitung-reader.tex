

% TexStudio Spellchecking Command
% !TeX spellcheck = de_DE_OLDSPELL

\mysection{Einleitung}

\hypertarget{Einführung Kreis}%
Mit diesem Reader will ich Ihnen meine Sicht auf Wissenschaft und Studium näherbringen
und durch ausgewählte Texte illustrieren. 

Meine Sicht? Ich schränke gleich wieder ein: \textit{Meine} Sicht auf Wissenschaft ist ziemlich unwichtig\footnote{Wir
sehen einmal von der Tatsache ab, daß ich in der Rolle als
Lehrer gelegentlich von der Gesellschaft die Aufgabe zugewiesen bekomme,
Sie zu bewerten. In diesem Moment wird meine Sicht für Sie
natürlich ziemlich wichtig sein.} -- wichtig wäre,
daß Sie sich selber \textit{Ihre eigene} Sicht auf Wissenschaft erarbeiten,
um dann über Instrumente zu verfügen, mit denen Sie Ihre Sicht anderen
erklären, sie begründen und gegen kritische Argumente verteidigen\footnote{\textit{Verteidigen} bedeutet nicht,
daß Sie lernen sollen, Ihre Sicht rhetorisch, populistisch oder gar stur durchzusetzen. 
Im Gegenteil: Die wissenschaftliche Position umfaßt immer
höchste Skepsis gegenüber der eigenen Position und die Bereitschaft, anderen Argumenten aufzunehmen, um diese dann
zu entkräften, um sie zu übernehmen oder um auf spätere Abklärung zu warten.} können. 
Das ist das Ziel dieses Readers.

\textit{Ihre} Sicht auf Wissenschaft wird sich möglicherweise von der Sicht
anderer Menschen unterscheiden -- oder aber mit ihr übereinstimmen. Was
von beidem der Fall ist, hat jetzt nicht die allergrößte 
Bedeutung. Wichtiger ist, daß \textit{Sie} wissen, \textit{warum} Ihre Sicht so-und-so ist und
\textit{warum} Sie diese für sinnvoll halten\footnote{Ich muß das in der Fußnote gleich wieder einschränken: 
Sicher kennen Sie den Witz mit dem Geisterfahrer auf der Autobahn. 
Bob hört im Autoradio die Nachricht: \textit{Achtung, Achtung: Auf
der A7 kommt Ihnen ein Geisterfahrer entgegen.} Neugierig schaut Bob auf die
Straße und er ruft aus: \enquote{\textit{Ein} Geisterfahrer? Nein! Hunderte! Tausende!!}
Der Witz soll uns darauf aufmerksam machen, daß eine Sichtweise, die sich von der Sichtweise sehr vieler
Menschen stark unterscheidet, nicht allein deshalb hilfreich, wichtig oder gar \enquote{richtig} ist, weil es
\textit{Ihre eigene} Sichtweise ist. Das könnte man dem oben geschriebenen Text nämlich -- fälschlicherweise --
entnehmen.
Andererseits war \textit{jede} wichtige wissenschaftliche Neuerung irgendwann einmal
die fixe Idee einer \enquote{verrückten} Einzelperson (solche Fälle sind aber heute historisch ausgesprochen selten).
Ihre Begründung, Ihr \textit{warum}, muß also Antworten enthalten auf die vielen Rückfragen
anderer Menschen -- und \textit{das}, dieser intersubjektive Diskurs, macht Wissenschaft aus.
}. Wird Ihre Sicht im Diskurs mit anderen dann brüchig, so würden Sie Ihre Sicht nachkorrigieren;
umgekehrt würde auch Ihr Gesprächspartner seine Sicht nachjustiieren, wenn Sie ihm überzeugende
Argumente darlegen\footnote{Der Satz schildert den Idealfall!}. 



Natürlich wäre es sinnvoll, wenn Sie sich jetzt 2 Jahre Zeit nehmen, in denen Sie
die 100 wichtigsten Texte zum Thema Wissenschaft lesen könnten. 
Sie würden sich mit Wissenschaftsgeschichte befassen und dabei an verschiedenen
Beispielen sehen, warum welche Ansätze zum Ziel geführt haben und welche
fehlgeschlagen sind. Sie werden dazu auch 100 Texte benötigen, denn immerhin betreibt die Menschheit
mindestens rund 5.000 Jahre Wissenschaft und hat in dieser Zeit sehr viel gelernt.

Diese Zeit
haben Sie aber nicht -- und vermutlich wollen Sie sich diese Zeit auch nicht nehmen:
Sie wollen Informatiker werden und nicht Wissenschafts\-theoretiker!

\paragraph{Checkliste \enquote{Wissenschaft}:}
Warum also gebe ich Ihnen nicht gleich einfach jene Checkliste, auf welcher die
wichtigsten 122 Regeln für wissen\-schaft\-li\-ches Arbeiten zusammengefaßt sind? Diese 122 Regeln
lernen
Sie dann auswendig und alles ist gut.

122 Regeln? Ist das nicht etwas viel?

Eigentlich ist die Zahl sehr klein! Sie entspricht der Kapitelanzahl
des schönen Buches \textit{Harry Potter and the Methods of Rationality}, das
ich Ihnen gleich als ersten der 100 Texte empfehlen würde.\footnote{Der Text 
ist übrigens \href{http://www.hpmor.com/}{\online} frei zugänglich und ich
würde ihn in der Tat als einen von 100 Texten empfehlen.}

\paragraph{7 Regeln:}
Sicher wären Ihnen 7 Regeln lieber! Die
Psychologen und Päda\-go\-gen bestätigen, daß 7 gerade jene Zahl von Regeln ist, die
sich ein Mensch einfach und in einem Anlauf gerade noch merken kann\footnote{\textbf{Wichtig:} Daher stammt auch die Regel, daß auf einer Folie
nicht mehr als 7 verschiedene Punkte stehen sollten und ein Vortrag höchstens 7 --- besser weniger! --- \enquote{take home messages}
enthalten soll.}.
Also lieber 7 Regeln! 

Und was sind diese 7 Regeln nun?

\paragraph{Wissenschaft als Regelwerk?}
Leider klappt das so nicht. Wissenschaft läßt sich
\textit{gerade nicht} auf einen kleinen Satz von Regeln reduzieren,
die man dann \textit{Methodik} nennt und einfach abarbeitet.
Die Gründe dazu finden Sie im ersten Text dieses Readers, in dem schönen
Vortrag von \textsc{Richard Feynman} über \textit{Cargo Cult Wissenschaft}.

Für Ungeduldige hier \hyperlink{Ank:Cargo Cult Wissenschaft}{ein Link auf die deutsche Version}
und \hyperlink{Ank:Cargo Cult Science}{auf die englische Version}.

\textit{Andererseits} kann Logik gerade als Versuch betrachtet
werden, das gesamte menschliche Nachdenken auf eine möglichst kleine
Anzahl von Regeln zu reduzieren. Wie Sie in der Logik-Vorlesung
gesehen haben: Mit 7 Regeln kommen wir schon sehr weit.
Ist also die Position von \textsc{Feynman} falsch? 
Gibt es also diese 7 Regeln -- und dann landen die Flugzeuge?

\textit{Eine weitere Sichtweise} entstammt der Empirie und reduziert
Wissenschaft auf das mathematisch-statistische Analysieren gemessener Daten
und Fakten. Wir leben in einer Zeit, in der unsere Rechner dieses auf neue und
uns daher beeindruckende Weise beherrschen. Deshalb gilt diese Herangehensweise (\enquote{data science}) 
heute auch als modern. In der Tat: Die Ergebnisse von \enquote{big data} sind beeindruckend.
Man kann argumentieren, daß diese Art von Wissenschaft zur kompaktesten Beschreibung des gesamten
Wissensbestandes führt, der in einer Datensammlung enthalten ist.
Dieser Ansatz findet aber auch seine Kritiker. So schreibt beispielsweise der Wissenschaftsphilosoph
\textsc{Konrad Liessmann}\footnote{K. Liessmann: Bildung als Provokation. Zsolnay Verlag, 2017. p216.}
sehr kritisch über rein empirische Ansätze.

{
\begin{quote}
\enquote{Die neue Liebe zu den Fakten ist verräterisch. Intellektuelle haben Fakten immer mißtraut,
weil sie um deren Kontextabhängig\-keit wußten. Empirie, so formulierte es einmal mit
unangenehmer Schärfe der Philosoph \textsc{Günther Anders}, ist nur etwas für Idioten. Denen
mangelt es nämlich an der Fähigkeit, über das Handgreifliche hinaus zu denken. Solche
Beschränktheit mag auch bei so manchen Rufen nach einer Zensur des Netzes und nach
automatisierten Fakten-Checks eine Rolle spielen.}
\end{quote}
}

\paragraph{Fazit?}
Die Frage ist nun \textit{nicht}, welche der vielen\footnote{Wir haben unser hier nur einige, wenige
ausgewählte
Beispiele angesehen.} Positionen \enquote{richtig} ist -- denn das ist nicht so einfach zu
beantworten\footnote{Sprechen Sie allerdings mit einem überzeugten Proponenten einer
Sichtweise, so werden Sie oft eine andere Antwort erhalten und viele Argumente, warum gerade diese oder
jene Sichtweise einfach richtig sein muß.}. Die Antwort ist aber \textit{ebenso nicht} ein
skeptisches oder hilfloses
\enquote{Kommt darauf an, alles ist relativ und wir wissen nichts.}
Der Erfolg von Natur- und Ingenieurwissenschaften, wie groß oder klein wir ihn nun einschätzen,
und unabhängig davon, wie wir ihn bewerten, ist unmittelbar greifbar.

Wissenschaft bewegt sich zwischen diesen Positionen.

Genau deshalb habe ich
Ihnen diesen Reader zusammengestellt. 
Er soll Ihnen dabei helfen, im Laufe des Semesters unterschiedliche
Aspekte von 
Wissenschaft und unterschiedliche Sichten auf wissenschaftliches
Arbeiten zu erfahren und daraus dann Ihre eigene Sicht zu entwickeln.
Für alle Texte gilt übrigens, wie auf Twitter: \textit{Citing does not mean endorsement}.
Der Prozeß, aus Materialien selber gedanklich etwas zu entwickeln ist wesentlich effizienter
als das Reinfressen-Rauskotzen des \enquote{Bulimie-Lernens}, auch wenn er \textit{zunächst} einmal mehr
inneres Engagement erfordert.

\subsection*{Muß ich das alles lesen? Ist das prüfungsrelevant?}

Zu Studienbeginn haben Sie sicherlich den Satz gehört \enquote{Es gibt keine dummen Fragen.}

\textit{Diese} zwei Fragen allerdings sollten Sie nicht stellen. Als  Sie Ihr Studium begannen,
haben Sie diese bereits für sich beantwortet.
Studium (lateinisch für \textsc{studere}, nach etwas streben, sich um etwas bemühen)
bedeutet ja, daß Sie sich eifrig strebend darum bemühen \textit{wollen}, in Ihrer Fachdisziplin das
Beste zu geben, was Sie können. 

Gleichwohl bleibt diese Antwort unbefriedigend. Jeder Wissenschaftler weiß um den Wert
und die Knappheit von Zeit, hat daher Verständnis, daß das Angestrebte nicht
immer umsetzbar ist, strebt aber auch immer danach, sich weiterzuentwickeln
und mehr zu lernen. 

Es ist daher eine wesentliche Kompetenz wissenschaftlichen Arbeitens, für sich selber Prioritäten zu
setzen und aus diesen dann Handlungen abzuleiten. Das umfaßt auch Entscheidungen, mit welchen
Materialien man sich in welcher Tiefe beschäftigt.
Die Diskussionen in den begleitenden Veranstaltungen sollen helfen,
hier Akzente zu setzen; der Selbstreflexionsteil der Hausarbeit zielt auch auf diese Thematik ab.

\subsection*{Urheberrechtlicher Hinweis}

Die Weitergabe der Texte geschieht hier nach \S 60a UrhG. Dazu verweise ich auf eine
\href{https://www.ub.uni-koeln.de/USB/ilias/e-sem.pdf}{Handreichung der Universität
Köln} und auf \href{https://www.ub.uni-rostock.de/lernen-arbeiten/wissenschaftliches-arbeiten/urheberrecht/}{Hinweise der Universität Rostock}.

Voraussetzung für die Weitergabe ist der auf die Teilnehmer an einer Lehrveranstaltung eingeschränkte
Zuhörerkreis.
Das für den Zugriff auf die Datei notwendige Paßwort erhalten Sie im eingeschränkten Bereich auf StudIP
der Lehrveranstaltung \enquote{Informatik und Wissenschaft} sowie
\enquote{Informatik, Wissenschaft und Gesellschaft} der Universität Rostock
im jeweils aktuellen Semester.

\subsection*{Wie nutzen Sie diesen Reader am Besten?}

Besorgen Sie sich die Zip-Datei mit den Materialien dieses Readers 
und das Paßwort dazu auf StudIP.
Entpacken Sie die Zip-Datei in dasselbe Verzeichnis, in dem sich auch diese PDF-Datei des
Kommentars befindet. Nun können Sie aus der PDF-Datei des Kommentars direkt in die Materialien
des Readers navigieren, wenn Sie einen PDF-Reader nutzen, der Hyperlinks unterstützt.

