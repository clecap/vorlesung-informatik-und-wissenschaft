
% TexStudio Spellchecking Command
% !TeX spellcheck = de_DE_OLDSPELL

\unit{Schreiben auf Englisch}{Schreiben auf Englisch}

Wissenschaftliches Schreiben in der eigenen Muttersprache
ist anspruchsvoll.
Für die allermeisten Wissenschaftler ist es aber notwendig, in einer
fremden Sprache zu schreiben. Bis in das 18.\ Jahrhundert und somit mehr als
zweitausend Jahre lang war Latein die Sprache der Wissenschaft.
In den Natur- und Ingenieurwissenschaften folgte dann eine Phase, in welcher
Französisch, Englisch und Deutsch die weltweit wichtigsten Sprachen der
Wissenschaft waren. In der Informatik ist heute das Englische (fast) die einzige
Sprache (wichtiger) wissenschaftlicher Kommunikation.

Dieser Reader kann einen Kurs in Wissenschaftsenglish nicht ersetzen. Da Wortwahl und
Satzstruktur ein häufiges Problem darstellen, soll der folgende Text dabei helfen.

\add{Writing in English} 
{Jen Tsi Yang: An Outline of Scientific Writing for Researchers
with English as a Foreign Language.
World Scientific, Singapore, 1995. ISBN 9810224664.
Chapter 1: Word Choice and Chapter 2: Sentence Structure. pp~1--25.}
{FILES/yang-an-outline-of-scientific-writing-chap-1-and-2.pdf}
{}
\pages{25}

Das Wörterbuch\footnote{Eigentlich ist es kein Wörterbuch im \textit{klassischen} Sinn sondern ein 
\href{https://d-nb.info/1210187280/04}{Katalog}
von Textbausteinen, der nach Anwendungssituationen zweckmäßig sortiert ist. Gerade diese
Strukturierung macht diesen Text so wertvoll.
} 
von \textsc{Dirk Siepmann} liefert eine ausgesprochen hilfreiche Übersicht mit
den wichtigen Wendungen der englischen \textit{und} der deutschen Wissenschaftssprache in
einem übersichtlichen Nebeneinander. Es darf daher als Empfehlung für alle dienen,
die Formulierungsvorschläge in einer der beiden Sprachen suchen:

\add{Wörterbuch}
{D. Siepmann: Wörterbuch der allgemeinen Wissen\-schafts\-sprache. Deutscher Hochschulverband, 2020.}
{}
{\href{https://d-nb.info/1210187280/04}{Inhaltsverzeichnis}}.

Darüber hinaus gibt es etliche online-Hilfsmittel, die Sie nutzen können:
\begin{enumerate}
\item \href{https://www.leo.org}{Leo das \textbf{Lexikon}: https://www.leo.org}
\item \href{https://www.deepl.com}{DeepL der \textbf{Übersetzer}: https://www.deepl.com}
\item \href{https://www.powerthesaurus.org}{Power Thesaurus für den \textbf{Wortschatz:} https://www.powerthesaurus.org}
\item \href{https://www.grammarly.com}{Grammarly für die \textbf{Grammatik:} https://www.grammarly.com}
\end{enumerate}
