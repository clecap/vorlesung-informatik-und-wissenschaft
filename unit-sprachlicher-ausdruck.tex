
% TexStudio Spellchecking Command
% !TeX spellcheck = de_DE_OLDSPELL

\unit{Sprachlicher Ausdruck}{Sprachlicher Ausdruck}

Mit dem sprachlichen Ausdruck ist das so eine Sache. Oft liest (oder schreibt) man etwas und
erst im Laufe der Zeit fällt einem auf, was man da für einen Unsinn geschrieben hat.
Mein Lieblings-Antibeispiel für Definitionen ist die folgende  Definition der Stoffmenge,
die sich im Chemielexikon \url{https://www.chemie.de} findet\footnote{Abgerufen am
27. März 2022 unter \href{https://www.chemie.de/lexikon/Stoffmenge.html}{https://www.chemie.de/lexikon/Stoffmenge.html}.}
und von führenden Suchmaschinen unter den ersten Treffern ausgegeben wird,
wenn man nach diesem Begriff sucht.


{
\begin{quote}
\enquote{Mit Stoffmenge wird die quantitative Mengenangabe für Stoffe, 
besonders in der Chemie, bezeichnet. 
Diese Stoffmenge ist dabei weder Masse, noch Teilchenzahl, 
sondern im Internationalen Einheitensystem (SI) durch willkürliche Vereinbarung 
als Basisgröße eigener Art festgelegt.}
\end{quote}
}

Auch die Definition der Wikipedia\footnote{\url{https://de.wikipedia.org/wiki/Stoffmenge}, abgerufen am
27. März 2022 als \url{https://de.wikipedia.org/w/index.php?title=Stoffmenge&oldid=221542405}.} ist 
nicht viel besser:

{
\begin{quote}
\enquote{Die Stoffmenge (veraltet Molmenge oder Molzahl) mit dem Formelzeichen $n$ 
ist eine Basisgröße im Internationalen Einheitensystem (SI) und gibt \textit{indirekt} 
die Teilchenzahl einer Stoffportion an.}
\end{quote}
}

Das Wort \underline{Stoff}-\underline{Menge} wird als \underline{Menge} von 
\underline{Stoff}, also durch sich selber und somit gar nicht erklärt.
Das Attribute \enquote{indirekt} in der Wikipedia ruft beim aufmerksamen Leser nur Fragezeichen 
hervor (was wäre denn eine \enquote{direkte} Angabe?). Vergleicht man die beiden Definitionen,
so lernt man in der ersten, daß die Stoffmenge \enquote{nicht die Teilchenzahl ist}, und in der zweiten,
daß sie \enquote{indirekt die Teilchenzahl angibt}.

Wer also den Begriff der Stoffmenge nicht bereits kennt, wird ihn durch diese Definition auch nicht 
 erlernen. Die knappe und selbstbezügliche Formulierung verdeckt, daß es historisch ein weiter Weg war: Von \textsc{Demokrit}
bis zu \textsc{Dalton} mußte die Modellvorstellung von Materie reifen, bis ein
Konzept wie jenes der Stoffmenge überhaupt sinnvoll erschien. Noch im Jahr 1905 schrieb 
\textsc{Albert Einstein} seine Dissertation über eine thematisch sehr verwandte Fragestellung!

Ein treffender, sprachlicher Ausdruck erfordert also \textit{zunächst} einmal konzeptuelle Klarheit.
Anschließend müssen die Konzepte noch in die richtigen Worte gefaßt werden.
Beides sind anspruchsvolle Aufgaben!

In der Informatik können wir uns an einem schönen Text von \textsc{Peter Rechenberg}
orientieren, der auch eine ganze Reihe hilfreicher Übungen enthält. Ich verweise hier auf die
zwei wichtigsten Abschnitte und empfehle die weiteren Teile des Buches.

\add{Technisches Schreiben (1)}
{P. Rechenberg: Technisches Schreiben (nicht nur) für Informatiker.
3. Auflage 2006, Hanser-Verlag München, ISBN 3-446-40695-6.
Kapitel 2: Klarheit, Kürze, Klang, pp. 17-50.}
{FILES/rechenberg-technisches-schreiben-kapitel-2-klarheit-kuerze-klang.pdf}
{}
\pages{34}

\add{Technisches Schreiben (2)}
{P. Rechenberg: Technisches Schreiben (nicht nur) für Informatiker.
3. Auflage 2006, Hanser-Verlag München, ISBN 3-446-40695-6.
Kapitel 3: Einfachheit, pp. 51-77.}
{FILES/rechenberg-technisches-schreiben-kapitel-3-einfachheit.pdf}
{}
\pages{27}


