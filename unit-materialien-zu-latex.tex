
% TexStudio Spellchecking Command
% !TeX spellcheck = de_DE_OLDSPELL

\unit{Materialien zu \LaTeX}{Materialien zu Latex}

\paragraph{\LaTeX\ lernen -- aber wie?}
\LaTeX\ kann alles\footnote{\TeX\ ist Turing vollständig. Wir können in \TeX\ also auch Fibonacci-Zahlen berechnen,
Matrizen multiplizieren und den Sinus auf 10 Dezimalstellen genau bestimmen.}. Ohne Macro-Pakete verfügt \TeX\
bereits über mehr als 500 Programmierprimitiva. Die kann und soll man nicht
alle auf Vorrat lernen. Wichtig aber ist:

\begin{enumerate}
\item die Grundfunktionen zu kennen und anhand eines Beispiels einmal \textit{selber aktiv} ausprobiert zu haben.
\item zu wissen, \textit{wie} und \textit{wo} man sich bei einer bestimmten Anforderung Hilfe suchen kann.
\end{enumerate}

\paragraph{Regelmäßig üben:}
Wir werden in dieser Lehrveranstaltung immer wieder kleine Elemente von \LaTeX\ gemeinsam ansehen und üben.
Machen Sie bitte regelmäßig mit! Das ist wichtig, damit Sie Ihre Fragen und Probleme mit in die Diskussion
einbringen können und wir gemeinsam die Lösungen finden können!

\paragraph{Quellen:} Es gibt eine Vielzahl exzellenter Lernmaterialien, Tutorien und Beispielsammlungen.

\begin{enumerate}
\item \textbf{Overleaf Materialien}
\begin{enumerate}
\item \href{https://www.overleaf.com/learn/latex/Learn_LaTeX_in_30_minutes}{Learn \LaTeX\ in 30 minutes}
\item \href{https://www.overleaf.com/learn/latex/LaTeX_video_tutorial_for_beginners_(video_1)}{\LaTeX\ video tutorial for beginners}
\item Das Menu auf der linken Seite der \href{https://www.overleaf.com}{Overleaf Webseite} bietet zu fast jedem Thema von \LaTeX\ ein
Tutorial mit ausführlichen Beispielen. 
\end{enumerate}
\item \textbf{Learnlatex.org}
\begin{enumerate}
\item \href{https://www.learnlatex.org/de/}{\LaTeX\ lernen mit Learnlatex.org (deutsch)}
\item \href{https://www.learnlatex.org/en/}{Learning \LaTeX\ with Learnlatex.org (englisch)}
\end{enumerate}
\item \textbf{Online Materialien auf Englisch}
\begin{enumerate}
\item \href{https://en.wikibooks.org/wiki/LaTeX}{\LaTeX\ Wiki Book}
\item \href{https://tobi.oetiker.ch/lshort/lshort.pdf}{{\LaTeX}2e in 139 minutes}
\item \href{https://www.dickimaw-books.com/latex/novices/novices-report.pdf}{\LaTeX\ for Complete Novices}
\end{enumerate}
\pagebreak%
\item \textbf{Online Materialien auf Deutsch}
\begin{enumerate}
\item \href{https://tu-dresden.de/mn/math/stochastik/das-institut/beschaeftigte/jan-rudl/ressourcen/dateien/latex_win/LaTeX-Kurs.pdf?lang=en}{Einführung in \LaTeX\ der TU Dresden}

\item \href{http://www.nagel-net.de/Latex/DOKU/Latexkurs_Skript.pdf}{Einführung in \LaTeX\ der Universität Tübingen}
\end{enumerate}

\item \textbf{FAQs:}
\begin{enumerate}
\item \href{https://texfaq.org/}{Texfaq}
\item \href{https://tex.stackexchange.com/}{Stackexchange}
\end{enumerate}

\item \textbf{Beispiele}
\begin{enumerate}
\item \href{https://texample.net/tikz/examples/all/}{Sammlung von Beispielen mit dem Zeichenpaket \tikz\ / \textsc{pgf}}\footnote{\tikz\ / \textsc{pgf} ist
ein extrem umfangreiches Makropaket für Graphiken aller Art. Der beste Weg, es zu lernen, ist anhand von Beispielen,
die nahe an der eigenen Anforderung liegen.}
\item \href{https://www.mlte.de/latex/beispiele/}{\LaTeX\ Beispiele von Malte}
\item \href{https://docs.freitagsrunde.org/Veranstaltungen/techtalk/2016/slides-plotting-2016-02-12.pdf}{Einführung in \tikz\ mit Beispielen}
\item \href{https://github.com/clecap/latex-examples}{latex-examples} Github repository von Clemens Cap.
\end{enumerate}
\item \textbf{\LaTeX-Beamer:} Dokumente mit unterschiedlichem Detailgrad\label{latexbeamermaterial}
\begin{enumerate}
\item \href{https://www.overleaf.com/learn/latex/Beamer}{Kurzeinführung auf Overleaf}
\item \href{https://www.overleaf.com/learn/latex/Beamer_Presentations\%3A_A_Tutorial_for_Beginners_(Part_1)\%E2\%80\%94Getting_Started}{5-teiliges Tutorial auf Overleaf}
\item \href{https://www.tp.nt.uni-siegen.de/talks/files/Feger_Journal-Club_13-03-2006.pdf}{Vortrag zu Beamer}
\item \href{https://statsoz-neu.userweb.mwn.de/lehre/2016_WiSe/Latex_Kurs/material/Beamer.pdf}{Einführungskurs zu Beamer}
\item \href{http://tug.ctan.org/macros/latex/contrib/beamer/doc/beameruserguide.pdf}{234-seitiger User Guide}\footnote{Die Idee
ist natürlich nicht, daß Sie jetzt diese 234 Seiten alle durcharbeiten. Lassen Sie sich einfach von simplen Beispielen
inspirieren, wie etwa \href{http://www2.informatik.uni-freiburg.de/~frank/latex-kurs/latex-kurs-3/Latex-Kurs-3.html}{hier}.}
\end{enumerate}


\item \textbf{Referenzhandbücher}

\begin{enumerate}
\item \href{https://texdoc.org/serve/latex2e.pdf/0}{\LaTeX\ Reference Manual (PDF)}
\item \href{https://tug.org/texinfohtml/latex2e.html}{\LaTeX\ Reference Manual (HTML)}
\item \href{http://tug.ctan.org/macros/latex/contrib/beamer/doc/beameruserguide.pdf}{Beamer Referenzhandbuch}
\item \href{https://mirror.informatik.hs-fulda.de/tex-archive/graphics/pgf/base/doc/pgfmanual.pdf}{\tikz / PGF Referenzhandbuch}\footnote{Pro Tipp: Es ist durchaus anspruchsvoll, in diesem über 1.000 Seiten langen Handbuch etwas zu finden. Mit ein wenig Übung gibt man bei Google
einige Suchstichworte ein, zusammen mit dem Schlüsselwort \texttt{TIKZ} -- und findet dann eine geeignete Stelle auf
Stackexchange, die genau das Gewünschte erklärt.}
\end{enumerate}
\end{enumerate}




