
% TexStudio Spellchecking Command
% !TeX spellcheck = de_DE_OLDSPELL


\unit{Vortragen}{Vortragen}

\label{vortragen}

Schreiben ist nur eine Form wissenschaftlicher Präsentation.
Der folgende Text gibt zunächst einige Anregungen zur Gestaltung einer Präsentation.

\add{Vortragen (1)}
{B. Heesen: Wissenschaftliches Arbeiten: Methodenwissen für das \mbox{Bachelor-,} Master- und Promotionsstudium.
3. Auflage. Springer-Verlag Berlin Heidelberg 2014. 
Kapitel 7: Präsentation wissenschaftlicher Erkenntnisse,
pp. 99-104.}
{FILES/heesen-wissenschaftliches-arbeiten-kapitel-7-praesentation-wissenschaftlicher-erkenntnisse.pdf}
{}
\pages{6}


Wie unterschiedlich das dann in der Praxis umgesetzt werden kann, soll nun diese
Übersicht zeigen:

\begin{enumerate}
\item \add{Vortragen (2)}
{F. Kindermann, Hinweise für den Seminarvortrag. Universität Regensburg.}
{FILES/kindermann-hinweise-fuer-einen-seminarvortrag.pdf}
{\href{https://www.uni-regensburg.de/assets/wirtschaftswissenschaften/vwl-kindermann/resources/hinweise_vortrag.pdf}{\online}}
\pages{2}.

\item \add{Vortragen (3)}
{U. Jäger: Bewertungskriterien für einen Seminarvortrag.}
{FILES/ulrike-jaeger-bewertungskriterien.pdf}
{\href{https://www.informatik.hu-berlin.de/de/forschung/gebiete/sam/Lehre/proseminar-sysml/material/bewertungskriterien-fur-einen-seminarvortrag}{\online}}
\pages{1}

\item \add{Vortragen (4)}
{F. Mattern: Seminarvortrag -- Hinweise zur Prä\-sen\-ta\-tion.
ETH Zürich, Februar 2010.}
{FILES/seminarvortraege.pdf}
{}
\pages{31}

\item \add{Vortragen (5)}
{G. Münster: Der goldene Weg zum perfekten Seminarvortrag. Institut für theoretische Physik,
Universität Münster.}
{FILES/der-goldene-weg-zum-perfekten-seminarvortrag}
{\href{https://www.uni-muenster.de/Physik.TP/archive/Seminare/anleitung.html}{\online}}
\pages{2}

\item Es gibt noch \href{https://duckduckgo.com/?q=Wie+halte+ich+einen+Seminarvortrag&ia=web}{ein klein wenig mehr dazu [Internet-Suche]}.
Sie werden also nicht umhin kommen, sich nun zu entscheiden, ob Sie genügend Materialien und
Anregungen erhalten haben -- und \textit{welchen individuellen
Charakter} Ihr Seminarvortrag tragen soll.
\end{enumerate}

Die Empfehlungen für die Informatik sehen dabei oftmals sehr ähnlich aus wie jene aus anderen Disziplinen.
So können Sie etwa aus diesem Merkblatt der Germanistik viele wichtige Hinweise entnehmen:

\add{Handreichung}
{H. Siebenpfeiffer: Präsenta\-tions\-formen \& Präsentationstechniken.
Handreichung des Instituts für Germanistik der Universität Greifswald.}
{FILES/Merkblatt-3-Praesentationsformen-Praesentationstechniken}
{\href{https://germanistik.uni-greifswald.de/storages/uni-greifswald/fakultaet/phil/germanistik/Mitarbeitende/Siebenpfeiffer/Merkblatt__3_Praesentationsformen___Praesentationstechniken.pdf}{\online}}
\pages{4}

