

% TexStudio Spellchecking Command
% !TeX spellcheck = de_DE_OLDSPELL


\mysection{Nachwort}

\paragraph{Textauswahl und Bias:}
Neben der Sichtweise, die dieser Reader durch die Auswahl seiner Texte
transportiert, gibt es natürlich noch viele weitere.

Jede Auswahl von Texten geschieht unter bestimmten Perspektiven. Dem
\textit{confirmation bias}\footnote{Siehe \hyperref[biases]{Kurzerklärung im Reader} oder \href{https://catalogofbias.org/biases/confirmation-bias/}{Bias Catalog, Confirmation Bias}.} und dem 
\textit{selection bias}\footnote{Siehe \hyperref[biases]{Kurzerklärung im Reader} oder
\href{https://catalogofbias.org/biases/selection-bias/}{Bias CatalogSelection Bias}.} 
ist nicht so leicht zu entkommen. Eine Diskussion \textit{aller}
oder auch nur \textit{aller wichtigen} Zugänge zu Wissenschaft braucht dann aber doch
wieder 100 Texte. Hier schließt sich somit der Kreis zur \hyperlink{Einführung Kreis}{Einführung} in diesen Reader,
der durch meine Textauswahl doch wieder meine Sicht auf Wissenschaft in den Vordergrund stellt.

